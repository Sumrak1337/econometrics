\section{Постановка задачи}

Необходимо провести однофакторный дисперсионный анализ приведённых ниже данных об условии хранения продуктов и содержании влаги. При необходимости применить метод линейных контрастов.

\VerbatimInput{figures/data.txt}

На уровне значимости $\alpha = 0.05$ проверить гипотезу о том, что условия хранения продукта не оказывают влияния на содержание влаги.

\section{Ход работы}

Цель дисперсионного анализа -- исследование наличия или отсутствия существенного влияния какого-либо качественного или количественного фактора на изменения исследуемого результативного признака. Для этого фактор разделяют на классы градации и выясняют, одинаково ли влияние фактора путём исследования значимости между средними в наборах данных.

Однофакторный дисперсионный анализ основан на том, что сумму квадратов отклонений статистического комплекса возможно разделить на состовляющие компоненты:

\begin{equation}
	SS = SS_a + SS_e
\end{equation}
или

\begin{equation}\label{disps}
	\sum\limits_{r=1}^k \sum\limits_{i=1}^{n_i} \left(X_{ir} - \overline{X} \right)^2 = \sum\limits_{r=1}^k n_r \cdot (\overline{X_r} - \overline{X})^2 + \sum\limits_{r=1}^k \sum\limits_{i=1}^{n_i} \left(X_{ir} - \overline{X_r}\right) = Q_1 + Q_2 = Q,
\end{equation}
где 

\begin{equation}
	\overline{X_r} = \frac{1}{n_i} \cdot \sum\limits_{i=1}^{n_i} X_{ir}
\end{equation}
-- выборочное среднее;

\begin{equation}
	\overline{X} = \frac{1}{n} \cdot \sum\limits_{r=1}^k \sum\limits_{i=1}^{n_i} X_{ir}
\end{equation}
-- общее выборочное среднее.

Результаты исходных данных представляют собой $k$ независимых выборок с допущением, что они получены из $k$ нормально распределенных генеральных совокупностей, где $k=3$:

\begin{equation}
	\begin{matrix}
		X_{11}, \ldots, X_{1n_1} & N(m_1; \sigma^2) \\
		X_{21}, \ldots, X_{2n_2} & N(m_2; \sigma^2) \\
		\ldots & \ldots \\
		X_{k1}, \ldots, X_{kn_k} & N(m_k; \sigma^2) \\
	\end{matrix}
\end{equation}

Сформулируем нулевую гипотезу $H_0: m_1 = m_2 = \ldots = m_k$ и альтернативную $H_1:$ \textit{отрицание нулевой гипотезы}. По формуле (\ref{disps}) можно найти сумму квадратов отклонений, объяснённых влиянием фактора $k$, сумму квадратов отклонений ошибки и общую сумму квадратов.

Статистика $\frac{Q_2}{\sigma^2}$ распределена по закону $\chi^2$ с $n - k$ степенями свободы. При условии справедливости нулевой гипотезы статистика $\frac{Q_1}{\sigma^2}$ распределена по закону $\chi^2$ с $k - 1$ степенями свободы. Следовательно, статистика

\begin{equation}
	\frac{\frac{Q_1}{k - 1}}{\frac{Q_2}{n - k}}
\end{equation}
распределена по закону Фишера с $k - 1$, $n - k$ степенями свободы. При справедливости альтернативной гипотезы статистика имеет тенденцию принимать большие значения, так как числитель в среднем оказывается больше, чем при справедливости нулевой гипотезы.

Было найдено наблюдаемое и критическое значения $F$ критерия. Нулевая гипотеза $H_0$ отвергается при уровне значимости $\alpha = 0.05$, так как $F > F_crit$ (см. результаты в конце файла).

В случае отклонения нулевых гипотез, необходимо проверить следующие при помощи \textit{метода линейных контрастов}:

\begin{equation}\label{H0}
	H^{(1)}_0: m_1 = m_2; H^{(2)}_0: m_1 = m_3; H^{(3)}_0: m_2 = m_3; H^{(4)}_0: \frac{1}{2} (m_1 + m_3) = m_2
\end{equation}

После проведения однофакторного дисперсионного анализа и по его итогу принятия альтернативной гипотезы важно узнать, какие именно математические ожидания значимо отличались, а какие равны. Линейный контраст $Lk$ определяется по формуле

\begin{equation}
	Lk = \sum\limits_{r=1}^k c_r m_r,
\end{equation}
где $c_r$ - задаваемые константы, причём $\sum\limits_{r=1}^k c_r = 0$

Оценка линейного контраста имеет следующий вид:

\begin{equation}
	\hat{Lk} = \sum\limits_{r=1}^k c_r \overline{X_r}
\end{equation}

Оценка дисперсии $\hat{Lk}$ вычисляется по формуле

\begin{equation}
	S^2_{Lk} = \sum\limits_{r=1}^k \frac{c^2_r}{n_r} \cdot \hat{\sigma}^2 = \frac{Q_2}{n - k} \cdot \sum\limits_{r=1}^k \frac{c^2_r}{n_r}
\end{equation}

Для сформулированных нулевых гипотез (\ref{H0}) о равенстве математических ожиданий применим соответствующие линейные контрасты:

\begin{equation}
	Lk_1 = m_1 - m_2; Lk_2 = m_1 - m_3; Lk_3 = m_2 - m_3; Lk_4 = \frac{1}{2} (m_1 + m_3) - m_2
\end{equation}
с коэффициентами $c_1 = 1, c_2 = -1, c_3 = 0$

Доверительный интервал для оценок контраста вычисляется по формуле

\begin{equation}
	\hat{Lk} \mp S_{lk} \cdot \sqrt{(k - 1) \cdot F_{1-\alpha}(k - 1, n - k)}
\end{equation}

Проведя вычисления, все 4 нулевые гипотезы отвергаются на уровне значимости 5\%, так как $0$ не принадлежит ни одному из полученных доверительных интервалов.

Результаты работы приведены ниже

\VerbatimInput{figures/file.txt}
