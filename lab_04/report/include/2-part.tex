\subsection{Критерий Вилкоксона}

\textit{Алгоритм критерия Вилкоксона}

\begin{enumerate}
	\item Выдвинуть нулевую гипотезу $H_0: F(\cdot) = G(\cdot)$. Сформулировать альтернативную гипотезу $H_1: F(\cdot) \ne G(\cdot)$ (двусторонняя альтернатива)
	\item Объединить выборки $X_{[n]}$ и $Y_{[m]}$ в общий вариационный ряд и проранжировать. Пусть ранги элементов выборки $Y_{[m]} = \{y_1, \ldots, y_m\}$ обозначаются через $r_1, \ldots, r_m$
	\item Вычислить статистику Вилкоксона по следующей формуле:
	
	\begin{equation}
		W(X_{[n]}, Y_{[m]}) = r1 + \ldots + r_m
	\end{equation}

	\item Задать уровень значимости $\alpha$
	\item Найти критическую область критерия:
	
	\begin{equation}
		\left[\frac{m(m + 1)}{2}, w_{\frac{\alpha}{2},m,n} \right] \cup \left[m(n + m + 1) - w_{\frac{\alpha}{2},m,n}, mn + \frac{m(m + 1)}{2} \right],
	\end{equation}
	где $w_{\frac{\alpha}{2},m,n}$ -- квантиль статистики Вилкоксона.

	\item Если численное значение статистики $W(X_{[n]}, Y_{[m]})$ попадает в критическую область, то нулевая гипотеза $H_0$ отвергается, в противном случае нет оснований её отвергнуть при уровне значимости $\alpha$.
\end{enumerate}

Результаты работы обоих критерий приведены ниже.

\VerbatimInput{figures/file.txt}