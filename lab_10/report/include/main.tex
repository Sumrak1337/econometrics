\section{Постановка задачи}

Следующая таблица содержит данные о товарообороте для магазинов трёх типов в течение четырёх лет после открытия магазина. В клетках стоят средние значения по группе магазинов.

\VerbatimInput{figures/data.txt}

На уровне значимости 5\% необходимо проверить нулевые гипотезы об отсутствии эффектов столбцов и строк.

\section{Ход работы}

В отличие от однофакторного дисперсионного анализа, в двухфакторном анализе значения наблюдений как в строках, так и в столбцах являются неоднородным.

Каждое наблюдение может быть представлено в виде

\begin{equation}
	x_{ij} = \mu + b_i + t_j + \epsilon_{ij}, i=\overline{1,n}; j=\overline{1,k}
\end{equation}

\begin{equation}
	\sum\limits_{i=1}^n b_i = 0; \sum\limits_{j=1}^k t_j = 0.
\end{equation}

$\epsilon_{ij}$ предполагаются независимыми, одинаково распределёнными, подчиняющиеся нормальному распределению $N(0, \sigma^2)$.

Можно показать, что общая вариация равна сумме вариаций строк, вариаций столбцов и остаточной вариации:

\begin{equation}\label{main}
	\sum\limits_{j=1}^k \sum\limits_{i=1}^n \left(x_{ij} - \overline{x}_{..}\right)^2 = k \sum\limits_{i=1}^n \left(\overline{x}_{i.} - \overline{x}_{..} \right)^2 + n \sum\limits_{j=1}^k \left(\overline{x}_{.j} - \overline{x}_{..}\right)^2 + \sum\limits_{j=1}^k \sum\limits_{i=1}^n \left(x_{ij} - \overline{x}_{i.} - \overline{x}_{.j} + \overline{x}_{..} \right)^2
\end{equation}
где

\begin{equation*}
	\overline{x}_{..} = \frac{1}{nk} \sum\limits_{i=1}^n \sum\limits_{j=1}^k x_{ij},
\end{equation*}

\begin{equation*}
	\overline{x}_{i.} = \frac{1}{k} \sum\limits_{j=1}^k x_{ij}
\end{equation*}

\begin{equation*}
	\overline{x}_{.j} = \frac{1}{n} \sum\limits_{i=1}^n x_{ij}
\end{equation*}

В рассматриваемой модели наблюдений предполагается, что взаимодействие между эффектами строк и столбцов отсутствует. Последнее слагаемое в (\ref{main}) 

\begin{equation}
	S_1 = \sum\limits_{j=1}^k \sum\limits_{i=1}^n \left(x_{ij} - \overline{x}_{i.} - \overline{x}_{.j} + \overline{x}_{..} \right)^2
\end{equation}
будет иметь $(n - 1)(k - 1)$ степеней свободы.

Сформулируем две нулевые гипотезы, которые заключаются в том, что влияние каждого из факторов отсутствует.

$H_0: b_1 = b_2 = \ldots = b_n = 0$ - влияние фактора строк отсутствует; альтернативная - $H_1:$ - не все $b_i$ равны нулю.

$H_0: t_1 = t_2 = \ldots = t_k = 0$ - влияние фактора столбцов отсутствует; альтернативная - $H_1:$ - не все $t_i$ равны нулю.

\begin{itemize}
	\item Если верна первая нулевая гипотеза, то модель наблюдений принимает вид
	
	\begin{equation}
		x_{ij} = \mu + t_j + \epsilon_{ij}
	\end{equation}
	
	\begin{equation*}
		S_R = k \sum\limits_{i=1}^n \left(\overline{x}_{i.} - \overline{x}_{..}\right)^2
	\end{equation*}
	будет иметь $(n - 1)$ степень свободы.
	
	При справедливости гипотезы $H_0: b_i = 0$ статистика
	
	\begin{equation}
		F_1 = \frac{\frac{S_R}{n - 1}}{\frac{S_1}{(n - 1)(k - 1)}}
	\end{equation}
	подчиняется распределению Фишера со степенями свободы $(n - 1)$ и $(n - 1)(k - 1)$.
	
	\item Если верна вторая нулевая гипотеза, то модель наблюдений принимает вид
	
	\begin{equation}
		x_{ij} = \mu + b_i + \epsilon_{ij}
	\end{equation}
	
	\begin{equation*}
		S_G = n \sum\limits_{j=1}^k \left(\overline{x}_{.j} - \overline{x}_{..}\right)^2
	\end{equation*}
	будет иметь $(k - 1)$ степень свободы.
	
	При справедливости гипотезы $H_0: t_j = 0$, то статистика
	
	\begin{equation}
		F_2 = \frac{\frac{S_G}{k - 1}}{\frac{S_1}{(n - 1)(k - 1)}}
	\end{equation}
	подчиняется распределению Фишера со степенями свободы $(k - 1)$ и $(n - 1)(k - 1)$.
	
	\item При справедливости альтернативной гипотезы в обоих случаях статистика имеет тенденцию принимать большие значения, так как числитель в среднем оказывается больше, чем при справедливости нулевой гипотезы. Применяя статистики $F_1$ и $F_2$ можно проверить каждую из нулевых гипотез.
\end{itemize}

Результаты работы двухфакторного дисперсионного анализа приведена ниже.

\VerbatimInput{figures/file.txt}