\section{Постановка задачи}

В таблице (в конце файла) приводятся сведения о пациентах (показатель ДЛТ, стадия $N$) и результат лечения (произошёл выход в ремиссию или нет).

Необходимо проверить, имеется ли связь между:

\begin{itemize}
	\item Показателем ДЛТ и ремиссией;
	\item Показателем $N$ и ремиссией.
\end{itemize}

\section{Ход работы}

Для каждой пары исследуемых признаков построены таблицы сопряжённости $2 \times 2$:

\VerbatimInput{figures/table.txt}

Сформулируем 2 нулевые гипотезы:

\begin{itemize}
	\item $H_0^{(1)}:$ между ДЛТ и ремиссией нет связи;
	\item $H_0^{(2)}:$ между $N$ и ремиссией нет связи.
\end{itemize}

Альтернативные гипотезы к обоим нулевым гипотезам будет отрицание этой самой нулевой гипотезы. По полученным таблицам сопряжённости можно проверить нулевые гипотезы, воспользовавшись точным критерием Фишера для таблиц $2 \times 2$.

Результаты применения критерия приведены ниже.

\VerbatimInput{figures/file.txt}

Таким образом, на уровне значимости 5\% ни ДЛТ, ни стадия $N$ не влияют на переход пациентами в ремиссию. Однако $p$-значение критерия Фишера для стадии $N$ довольно близко в критическому значению, что означает почти полное влияние стадии $N$ на переход в ремиссию.

Исходные данные представлены ниже.

\VerbatimInput{figures/data.txt}
