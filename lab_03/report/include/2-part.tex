\subsection{Критерий $\omega^2$}

\textit{Алгоритм критерия $\omega^2$ в случае сложной гипотезы о нормальности распределения генеральной совокупности}

\begin{enumerate}
	\item Выдвинуть нулевую гипотезу $H_0: F_{\xi}(\cdot) = F_0(\cdot, \theta)$. Сформулировать альтернативную гипотезу $H_1: F_{\xi}(\cdot) \ne F_0(\cdot, \theta)$;
	\item Задать уровень значимости критерия $\alpha$;
	\item Найти оценки $\hat{\theta} = (\overline{x}, s^2)$ неизвестных параметров распределения $\theta = (a, \sigma^2)$;
	\item Вычислить значение исправленной формы статистики $\hat{\omega}^2_n$ следующим образом:
	\begin{enumerate}
		\item По выборке $X_{[n]}$ построить эмпирическую функцию распределения $F_n(x)$ по формуле (\ref{ecdf})
		\item Определить $\omega^2_n$ по формуле 
		
		\begin{equation}
			\omega^2_n = \frac{1}{12n} + \sum\limits_{i=1}^n \left(F_0(x_i) - \frac{2i - 1}{2n} \right)^2
		\end{equation}
		
		\item Вычислить значение исправленной формы модифицированной статистики $\omega^2$ по формуле
		
		\begin{equation}
			\hat{\omega}^2_n = \omega^2_n \left(1 + \frac{0.5}{n} \right)
		\end{equation}
		
	\end{enumerate}
	
	\item Найти критическую область - интервал $(w_{1-\alpha}; \infty)$. Квантиль $w_{1 - \alpha}$ можно найти из таблицы ниже
	
	\begin{center}
		\begin{tabular}{|c|c|c|c|c|c|}
			\hline
			Модифицированная форма & 0.15 & 0.1 & 0.05 & 0.025 & 0.01 \\
			\hline
			$\omega^2_n \left(1 + \frac{0.5}{n} \right)$ & 0.091 & 0.104 & 0.126 & 0.148 & 0.178 \\
			\hline
		\end{tabular}
	\end{center}
	
	\item Если численное значение статистики $\hat{\omega}^2_n$ попадает в интервал $(w_{1 - \alpha}; \infty)$, то нулевая гипотеза $H_0$ отвергается, в противном случае нет оснований отвергнуть нулевую гипотезу при уровне значимости приближённо равном $\alpha$.
	
\end{enumerate}

Результаты работы данного критерия представлены ниже.

\VerbatimInput{figures/file.txt}