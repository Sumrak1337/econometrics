\section{Постановка задачи}

В таблице приведены данные об уровне производительности труда по экономике США

\VerbatimInput{figures/data.txt}

Необходимо проверить гипотезу о том, что медиана генеральной совокупности $\theta_0 = 1.5$ при двусторонней альтернативе $\theta \ne \theta_0$.

\section{Ход работы}

Были изучены разделы 2.5.5, 2.5.7 из книги Буре В.М., Парилина Е.М., Седаков А.А. <<Методы прикладной статистики в R и Excel>>.

Были найдены первые разности, сформирована выборка $y_i = x_i - \theta_0$, $i = \overline{1, n}$ без нулевых элементов, где $x_i$ -- элементы полученной выборки из первых разностей. Так как количество наблюдений в первых разностях равно 28, то мною было принято решение о реализации алгоритма для большой выборки.

Дальнейшие вычисления производились в соответствии с указанными в вышеописанных разделах алгоритмами.

\textit{Алгоритм критерия знаков для одной выборки большого объема}

\begin{enumerate}
	\item Выдвинуть нулевую гипотезу $H_0: \theta = \theta_0$ и сформулировать альтернативную гипотезу $H_1: \theta \ne \theta_0$
	\item Здаать уровень значимости $\alpha$
	\item Вычислить значение статистики $S^*(X_{[n]})$ по следующим формулам
	
	\begin{equation}\label{S}
		S^*(X_{[n]}) = \frac{S(X_{[n]}) - \frac{n}{2}}{\sqrt{\frac{n}{4}}},
	\end{equation}

	\begin{equation}
		S(X_{[n]}) = \sum\limits_{i=1}^n s(y_i),
	\end{equation}

	\begin{equation}
		s(y_i) = \left\{
		\begin{matrix}
			1, & y_i > 0 \\
			0, & y_i \leq 0
		\end{matrix}
		\right.
	\end{equation}
	
	Случайная величина $s(y_i)$ принимает 2 значения: 0 и 1. Согласно выдвинутой нулевой гипотезе, вероятность каждого их этих значений равна $0.5$. Таким образом, данная задача сводится к схеме испытаний Бернулли, а проверка нулевой гипотезы сводится к проверке новой нулевой гипотезы $H_0: p = 0.5$
	
	\item Найти критическую область
	
	\begin{equation}
		\left(-\infty; u_{\frac{\alpha}{2}} \right) \cup \left(u_{1 - \frac{\alpha}{2}}; \infty \right),
	\end{equation}
	где $u_{\frac{\alpha}{2}}$ и $u_{1 - \frac{\alpha}{2}}$ -- квантили стандартного нормального распределения уровней $\frac{\alpha}{2}$ и $1 - \frac{\alpha}{2}$.
	
	\item Если численное значение статистики (\ref{S}) попадает в критическую область, то нулевая гипотеза отвергается, в противном случае нет оснований её отвергнуть при уровне значимости приближенно равном $\alpha$.
	
\end{enumerate}

Результат работы данного алгоритма представлен ниже.

\VerbatimInput{figures/file.txt}