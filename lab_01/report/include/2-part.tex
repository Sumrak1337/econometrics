\subsection{Корреляция}

Были сформированы 3 выборки следующим образом:

\begin{enumerate}
	\item выборка со значениями равномерно распределённой случайной величины из отрезка $[0, 1]$;
	\item выборка, полученная из предыдущей путём нахождения квантилей нормального распределения со случайными значениями математического ожидания и стандартного отклонения;
	\item выборка, сформированная с помощью стандартных средств Python.
\end{enumerate}

В ходе работы были вычислены выборочные коэффициенты корреляции для всех трёх полученных выборок, была проверена статистическая значимость каждого коэффициента на уровне значимости 5\% с помощью $t_r$-статистики с числом степеней свободы $n-2$:

\begin{equation}
	t_r = \frac{r}{\sqrt{1 - r^2} \sqrt{n - 2}}
\end{equation} 

Также были вычислены выборочные ковариации, средние квадратические отклонения для выборок и построена корреляционная матрица $3 \times 3$. Результаты данной и предыдущей работ представлены ниже в качестве сформированного вывода в консоль Python.

Из полученных значений видно, что между первой и второй выборкой имеется сильная корреляция (равна почти 1), тогда как выборки 2, 3 и 1, 3 корреляционных связей почти не имеют. 

\VerbatimInput{figures/file.txt}
