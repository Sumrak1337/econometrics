\subsection{Критерий Фишера-Снедекора}

\textit{Алгоритм критерия Фишера-Снедекора}

\begin{enumerate}
	\item Выдвинуть нулевую гипотезу $H_0: \sigma^2_1 = \sigma^2_2$. Сформулировать альтернативную гипотезу $H_1: \sigma^2_1 \ne \sigma^2_2$
	\item Задать уровень значимости критерия $\alpha$
	\item Вычислить значение статистики по формуле
	
	\begin{equation}\label{F}
		F(X_{[n]}, Y_{[m]}) = \frac{\frac{s^2_1n}{n - 1}}{\frac{s^2_2m}{m - 1}} = \frac{\tilde{s}^2_1}{\tilde{s}^2_2},
	\end{equation}
	где $s^2_1$, $s^2_2$ -- выборочные дисперсии выборок $X_{[n]}, Y_{[m]}$
	
	\item Найти критическую область 
	
	\begin{equation}
		\left[0; F_{\frac{\alpha}{2}, n - 1, m - 1} \right) \cup \left( F_{1 - \frac{\alpha}{2}, n - 1, m - 1}; \infty \right],
	\end{equation}
	где $F_{\frac{\alpha}{2}, n - 1, m - 1}$ и $F_{1 - \frac{\alpha}{2}, n - 1, m - 1}$ -- квантили уровней $\frac{alpha}{2}$ и $1 - \frac{alpha}{2}$ распределения Фишера с $(n - 1)(m - 1)$ степенями свободы
	
	\item Если значение статистики (\ref{F}) попадает в критическую область, то нулевая гипотеза $H_0$ отвергается, в противном случае нет основания её отвергнуть при уровне значимости $\alpha$.
\end{enumerate}
