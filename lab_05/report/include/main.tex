\section{Постановка задачи}

В таблице приводятся сведения об экспорте и импорте Бельгии

\VerbatimInput{figures/data.txt}

Считая, что выборки первых разностей подчиняются нормальному распределению, на уровне значимости 5\% необходимо проверить равенство средних и дисперсий по критериям Стьюдента и Фишера-Снедекора. Необходимо проверить нормальность выборок по критерию Колмагорова для сложных гипотез.

\section{Ход работы}

Изначально были изучены разделы 2.4.1, 2.4.2, 2.4.5, 2.4.8 из книги Буре В.М., Парилина Е.М., Седаков А.А. <<Методы прикладной статистики в R и Excel>>.

Была произведена обработка данных: вычислены первые разности и исключены повторения. На основе полученных временных рядов получены следующие функции распределения:

\begin{figure}[H]
	\begin{minipage}[H]{0.49\linewidth}
		\begin{center}
			\includegraphics[width=\linewidth]{figures/Import}
		\end{center}
	\end{minipage}
	\hfill
	\begin{minipage}[H]{0.49\linewidth}
		\begin{center}
			\includegraphics[width=\linewidth]{figures/Export}
		\end{center}
	\end{minipage}
\end{figure}

Дальнейшие вычисления по критериям производились в соответствии с алгоритмами, описывающие критерии.

\subsection{Критерий Стьюдента}

\textit{Алгоритм критерия Стьюдента в предположении известных дисперсий}

\begin{enumerate}
	\item Выдвинуть гипотезу $H_0: a_1 = a_2$. Сформулировать альтернативную гипотезу $H_0: a_1 \ne a_2$
	\item Задать уровень значимости критерия $\alpha$
	\item Вычислить значение статистики по формуле
	
	\begin{equation}\label{phi1}
		\phi_1\left(X_{[n]}, Y_{[m]}\right) = \frac{\overline{x} - \overline{y}}{\sqrt{\frac{\sigma^2_1}{n} + \frac{\sigma^2_2}{m}}},
	\end{equation}
	которая подчиняется стандартному нормальному распределению, при условии, что справедлива гипотеза $H_0$.
	
	\item Найти критическую область
	
	\begin{equation}
		\left(-\infty; -u_{1 - \frac{\alpha}{2}} \right) \cup \left(u_{1 - \frac{\alpha}{2}}; \infty \right),
	\end{equation}
	где $u_{1 - \frac{\alpha}{2}}$ -- квантиль стандартного нормального распределения уровня $1 - \frac{\alpha}{2}$
	
	\item Если модуль численного значения статистики (\ref{phi1}) попадает в интервал $\left(u_{1 - \frac{\alpha}{2}}; \infty \right)$, то нулевая гипотеза $H_0$ отвергается, в противном случае нет оснований её отвергнуть при уровне значимости $\alpha$.
\end{enumerate}

\textit{Алгоритм критерия Стьюдента в предположениях неизвестных, но равных дисперсиях}

\begin{enumerate}
	\item Выдвинуть гипотезу $H_0: a_1 = a_2$. Сформулировать альтернативную гипотезу $H_0: a_1 \ne a_2$
	\item Задать уровень значимости критерия $\alpha$
	\item Вычислить значение статистики по формуле
	
	\begin{equation}\label{phi2}
		\phi_2\left(X_{[n]}, Y_{[m]}\right) = \frac{\overline{x} - \overline{y}}{\hat{s}\sqrt{\frac{1}{n} + \frac{1}{m}}},
	\end{equation}
	где
	
	\begin{equation}
		\hat{s} = \sqrt{\frac{ns^2_1 + ms^2_2}{n + m - 2}} = \sqrt{\frac{(n - 1) \tilde{s}^2_1 + (m - 1) \tilde{s}^2_2}{n + m - 2}},
	\end{equation}
	где $s^2_1$, $s^2_2$ -- выборочные дисперсии, $\tilde{s}^2_1$, $\tilde{s}^2_2$ -- исправленные выборочные дисперсии выборок $X_{[n]}, Y_{[m]}$
	
	\item Найти критическую область
	
	\begin{equation}
		\left(-\infty; -t_{1 - \frac{\alpha}{2}, n + m - 2} \right) \cup \left(t_{1 - \frac{\alpha}{2}, n + m - 2}; \infty \right),
	\end{equation}
	где $t_{1 - \frac{\alpha}{2}, n + m - 2}$ -- квантиль распределения Стьюдента с $(n + m - 2)$ степенями свободы уровня $1 - \frac{\alpha}{2}$
	
	\item Если модуль численного значения статистики (\ref{phi2}) попадает в интервал $\left(t_{1 - \frac{\alpha}{2}, n + m - 2}; \infty \right)$, то нулевая гипотеза $H_0$ отвергается, в противном случае нет оснований её отвергнуть при уровне значимости $\alpha$.
\end{enumerate}

\subsection{Критерий Фишера-Снедекора}

\textit{Алгоритм критерия Фишера-Снедекора}

\begin{enumerate}
	\item Выдвинуть нулевую гипотезу $H_0: \sigma^2_1 = \sigma^2_2$. Сформулировать альтернативную гипотезу $H_1: \sigma^2_1 \ne \sigma^2_2$
	\item Задать уровень значимости критерия $\alpha$
	\item Вычислить значение статистики по формуле
	
	\begin{equation}\label{F}
		F(X_{[n]}, Y_{[m]}) = \frac{\frac{s^2_1n}{n - 1}}{\frac{s^2_2m}{m - 1}} = \frac{\tilde{s}^2_1}{\tilde{s}^2_2},
	\end{equation}
	где $s^2_1$, $s^2_2$ -- выборочные дисперсии выборок $X_{[n]}, Y_{[m]}$
	
	\item Найти критическую область 
	
	\begin{equation}
		\left[0; F_{\frac{\alpha}{2}, n - 1, m - 1} \right) \cup \left( F_{1 - \frac{\alpha}{2}, n - 1, m - 1}; \infty \right],
	\end{equation}
	где $F_{\frac{\alpha}{2}, n - 1, m - 1}$ и $F_{1 - \frac{\alpha}{2}, n - 1, m - 1}$ -- квантили уровней $\frac{alpha}{2}$ и $1 - \frac{alpha}{2}$ распределения Фишера с $(n - 1)(m - 1)$ степенями свободы
	
	\item Если значение статистики (\ref{F}) попадает в критическую область, то нулевая гипотеза $H_0$ отвергается, в противном случае нет основания её отвергнуть при уровне значимости $\alpha$.
\end{enumerate}

\subsection{Критерий Колмогорова}

\textit{Алгоритм критерия солгасия Колмогорова в случае сложной гипотезы о нормальности распределения генеральной совокупности}

\begin{enumerate}
	\item Выдвинуть нулевую гипотезу $H_0: F_{\xi}(\cdot) = F_0(\cdot, \theta)$. Сформулировать альтернативную гипотезу $H_1: F_{\xi}(\cdot) \ne F_0(\cdot, \theta)$;
	\item Задать уровень значимости критерия $\alpha$;
	\item Найти оценки $\hat{\theta} = (\overline{x}, s^2)$ неизвестных параметров распределения $\theta = (a, \sigma^2)$;
	\item Вычислить значение исправленной формы статистики $\tilde{D^*_n}$ следующим образом:
	\begin{enumerate}
		\item По выборке $X_{[n]}$ построить эмпирическую функцию распределения $F_n(x)$ по формуле:
		
		\begin{equation}\label{ecdf}
			F_n(x) = \left\{
			\begin{matrix}
				0, & x < x_1, \\
				\frac{1}{n}, & x_1 \leq x < x_2, \\
				\vdots & \\
				\frac{k}{n}, & x_k \leq x < x_{k+1}, \\
				\vdots & \\
				1, & x \geq x_n
			\end{matrix}
			\right.
		\end{equation}
		
		\item Определить $D^*_n$ по формуле 
		
		\begin{equation}
			D^*_n = \max_{1 \leq i \leq n} \left\{\frac{i}{n} - F_0(x_i), F_0(x_i) - \frac{i - 1}{n} \right\}
		\end{equation}
		
		\item Вычислить значение исправленной формы модифицированной статистики Колмогорова по формуле
		
		\begin{equation}
			\tilde{D^*_n} = D^*_n \left(\sqrt{n} - 0.01 + \frac{0.85}{\sqrt{n}} \right)
		\end{equation}
		
	\end{enumerate}
	
	\item Найти критическую область - интервал $(d_{1-\alpha}; \infty)$. Квантиль $d_{1 - \alpha}$ можно найти из таблицы ниже
	
	\begin{center}
		\begin{tabular}{|c|c|c|c|c|c|}
			\hline
			Модифицированная форма & 0.15 & 0.1 & 0.05 & 0.025 & 0.01 \\
			\hline
			$D^*_n (\sqrt{n} - 0.01 + \frac{0.85}{\sqrt{n}})$ & 0.775 & 0.819 & 0.895 & 0.955 & 1.035 \\
			\hline
		\end{tabular}
	\end{center}
	
	\item Если численное значение статистики $\tilde{D^*_n}$ попадает в интервал $(d_{1 - \alpha}; \infty)$, то нулевая гипотеза $H_0$ отвергается, в противном случае нет оснований отвергнуть нулевую гипотезу при уровне значимости приближённо равном $\alpha$.
	
\end{enumerate}

Результаты работы всех критериев при уровнях значимости $\alpha = 0.05$ представлены ниже.

\VerbatimInput{figures/file.txt}

