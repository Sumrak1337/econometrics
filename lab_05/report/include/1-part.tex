\subsection{Критерий Стьюдента}

\textit{Алгоритм критерия Стьюдента в предположении известных дисперсий}

\begin{enumerate}
	\item Выдвинуть гипотезу $H_0: a_1 = a_2$. Сформулировать альтернативную гипотезу $H_0: a_1 \ne a_2$
	\item Задать уровень значимости критерия $\alpha$
	\item Вычислить значение статистики по формуле
	
	\begin{equation}\label{phi1}
		\phi_1\left(X_{[n]}, Y_{[m]}\right) = \frac{\overline{x} - \overline{y}}{\sqrt{\frac{\sigma^2_1}{n} + \frac{\sigma^2_2}{m}}},
	\end{equation}
	которая подчиняется стандартному нормальному распределению, при условии, что справедлива гипотеза $H_0$.
	
	\item Найти критическую область
	
	\begin{equation}
		\left(-\infty; -u_{1 - \frac{\alpha}{2}} \right) \cup \left(u_{1 - \frac{\alpha}{2}}; \infty \right),
	\end{equation}
	где $u_{1 - \frac{\alpha}{2}}$ -- квантиль стандартного нормального распределения уровня $1 - \frac{\alpha}{2}$
	
	\item Если модуль численного значения статистики (\ref{phi1}) попадает в интервал $\left(u_{1 - \frac{\alpha}{2}}; \infty \right)$, то нулевая гипотеза $H_0$ отвергается, в противном случае нет оснований её отвергнуть при уровне значимости $\alpha$.
\end{enumerate}

\textit{Алгоритм критерия Стьюдента в предположениях неизвестных, но равных дисперсиях}

\begin{enumerate}
	\item Выдвинуть гипотезу $H_0: a_1 = a_2$. Сформулировать альтернативную гипотезу $H_0: a_1 \ne a_2$
	\item Задать уровень значимости критерия $\alpha$
	\item Вычислить значение статистики по формуле
	
	\begin{equation}\label{phi2}
		\phi_2\left(X_{[n]}, Y_{[m]}\right) = \frac{\overline{x} - \overline{y}}{\hat{s}\sqrt{\frac{1}{n} + \frac{1}{m}}},
	\end{equation}
	где
	
	\begin{equation}
		\hat{s} = \sqrt{\frac{ns^2_1 + ms^2_2}{n + m - 2}} = \sqrt{\frac{(n - 1) \tilde{s}^2_1 + (m - 1) \tilde{s}^2_2}{n + m - 2}},
	\end{equation}
	где $s^2_1$, $s^2_2$ -- выборочные дисперсии, $\tilde{s}^2_1$, $\tilde{s}^2_2$ -- исправленные выборочные дисперсии выборок $X_{[n]}, Y_{[m]}$
	
	\item Найти критическую область
	
	\begin{equation}
		\left(-\infty; -t_{1 - \frac{\alpha}{2}, n + m - 2} \right) \cup \left(t_{1 - \frac{\alpha}{2}, n + m - 2}; \infty \right),
	\end{equation}
	где $t_{1 - \frac{\alpha}{2}, n + m - 2}$ -- квантиль распределения Стьюдента с $(n + m - 2)$ степенями свободы уровня $1 - \frac{\alpha}{2}$
	
	\item Если модуль численного значения статистики (\ref{phi2}) попадает в интервал $\left(t_{1 - \frac{\alpha}{2}, n + m - 2}; \infty \right)$, то нулевая гипотеза $H_0$ отвергается, в противном случае нет оснований её отвергнуть при уровне значимости $\alpha$.
\end{enumerate}
