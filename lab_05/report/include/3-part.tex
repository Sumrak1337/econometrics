\subsection{Критерий Колмогорова}

\textit{Алгоритм критерия солгасия Колмогорова в случае сложной гипотезы о нормальности распределения генеральной совокупности}

\begin{enumerate}
	\item Выдвинуть нулевую гипотезу $H_0: F_{\xi}(\cdot) = F_0(\cdot, \theta)$. Сформулировать альтернативную гипотезу $H_1: F_{\xi}(\cdot) \ne F_0(\cdot, \theta)$;
	\item Задать уровень значимости критерия $\alpha$;
	\item Найти оценки $\hat{\theta} = (\overline{x}, s^2)$ неизвестных параметров распределения $\theta = (a, \sigma^2)$;
	\item Вычислить значение исправленной формы статистики $\tilde{D^*_n}$ следующим образом:
	\begin{enumerate}
		\item По выборке $X_{[n]}$ построить эмпирическую функцию распределения $F_n(x)$ по формуле:
		
		\begin{equation}\label{ecdf}
			F_n(x) = \left\{
			\begin{matrix}
				0, & x < x_1, \\
				\frac{1}{n}, & x_1 \leq x < x_2, \\
				\vdots & \\
				\frac{k}{n}, & x_k \leq x < x_{k+1}, \\
				\vdots & \\
				1, & x \geq x_n
			\end{matrix}
			\right.
		\end{equation}
		
		\item Определить $D^*_n$ по формуле 
		
		\begin{equation}
			D^*_n = \max_{1 \leq i \leq n} \left\{\frac{i}{n} - F_0(x_i), F_0(x_i) - \frac{i - 1}{n} \right\}
		\end{equation}
		
		\item Вычислить значение исправленной формы модифицированной статистики Колмогорова по формуле
		
		\begin{equation}
			\tilde{D^*_n} = D^*_n \left(\sqrt{n} - 0.01 + \frac{0.85}{\sqrt{n}} \right)
		\end{equation}
		
	\end{enumerate}
	
	\item Найти критическую область - интервал $(d_{1-\alpha}; \infty)$. Квантиль $d_{1 - \alpha}$ можно найти из таблицы ниже
	
	\begin{center}
		\begin{tabular}{|c|c|c|c|c|c|}
			\hline
			Модифицированная форма & 0.15 & 0.1 & 0.05 & 0.025 & 0.01 \\
			\hline
			$D^*_n (\sqrt{n} - 0.01 + \frac{0.85}{\sqrt{n}})$ & 0.775 & 0.819 & 0.895 & 0.955 & 1.035 \\
			\hline
		\end{tabular}
	\end{center}
	
	\item Если численное значение статистики $\tilde{D^*_n}$ попадает в интервал $(d_{1 - \alpha}; \infty)$, то нулевая гипотеза $H_0$ отвергается, в противном случае нет оснований отвергнуть нулевую гипотезу при уровне значимости приближённо равном $\alpha$.
	
\end{enumerate}

Результаты работы всех критериев при уровнях значимости $\alpha = 0.05$ представлены ниже.

\VerbatimInput{figures/file.txt}
