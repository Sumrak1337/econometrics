\section{Постановка задачи}

Необходимо проверить гипотезу однородности данных ниже с помощью критерия Краскела-Уоллиса на уровне значимости $\alpha = 0.05$

\VerbatimInput{figures/data.txt}

\section{Ход работы}

Критерий Краскела-Уоллиса применяется, когда нельзя сказать ничего определённого об альтернативах к нулевой гипотезе, и неизвестно, их какого распределения были взяты выборки.

Результаты данных составляют $k$ независимых выборок, взятых из распределений с непрерывными функциями распределения.

\begin{equation}
	\begin{matrix}
		X_{11}, \ldots, X_{1n_1} \\
		X_{21}, \ldots, X_{2n_2} \\
		\ldots \\
		X_{k1}, \ldots, X_{kn_k} \\
	\end{matrix}
\end{equation}

Предлагается упорядочить все величины $X_{ij}$ по возрастанию и обозначить ранг $r_{ij}$ у каждого числа $X_{ij}$ во всей совокупности.

Далее формулируется нулевая гипотеза $H_0:$ \textit{все $k$ выборок однородны, то есть являются выборками из одного и того же закона распределения}, а также формулируется альтернативная гипотеза $H_1:$ \textit{отрицание нулевой гипотезы}.

Пусть $\overline{R} = \frac{1}{n_i} \cdot \sum\limits_{j=1}^{n_i} r_{ij}$ - средний ранг по выборке. Предполагается, что если между выборками нет систематических различий, то введённые средние ранги не должны отличаться от общего среднего ранга, рассчитанного по формуле $\frac{n + 1}{2}$, где $n = \sum\limits_{i=1}^k n_i$.

Статистика данного критерия имеет вид

\begin{equation}
	H = \frac{12}{n(n + 1)} \cdot \sum\limits_{i=1}^k n_i \cdot \left(\overline{R}_i - \frac{n + 1}{2}\right)^2.
\end{equation}

Критическое значение вычисляется из распределения $\chi^2$. Гипотеза $H_0$ отвергается на уровне значимости $\alpha$, если статистика критерия $H > \chi^2_{1 = \alpha}$, где $\chi^2_{1 = \alpha}$ -- квантиль уровня $1 - \alpha$ распределения хи-квадрат с $k - 1$ степенями свободы, в противном случае нет основания отвергнуть её на уровне значимости $\alpha$.

Из полученных результатов можно заключить, что все выборки были взяты из разных генеральных совокупностей, распределённых по разным законам распределения.

Результаты рабоыт критерия приведены ниже.

\VerbatimInput{figures/file.txt}
